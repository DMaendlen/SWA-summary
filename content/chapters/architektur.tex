\section{Architekturmuster}\label{architekturmuster}

\section{Unterschied Architektur- zu
Entwurfsmuster}\label{unterschied-architektur--zu-entwurfsmuster}

Architekturmuster beziehen sich auf das Gesamtsystem (grundlegendes
Problem), während sich Entwurfsmuster auf die Teilgebiete und die
Umsetzung beziehen (lokales Problem)

Beispiel: Einkommensgenerierung. Architektur: Gründung Escort-Service.
Entwurf: Standort, Personal, Verwaltung.

\section{Layer}\label{layer}

Software ist in logischen Schichten organisiert. Problem:

\begin{itemize}
\itemsep1pt\parskip0pt\parsep0pt
\item
  Unterteilung komplexer Systeme (Abhängigkeitsminimierung)
\item
  Horizontale Strukturierung (Obere Schicht = Client, Untere Schicht =
  Server)
\item
  Jede Schicht repräsentiert eine Abstraktion
\end{itemize}

Lösung:

\begin{itemize}
\itemsep1pt\parskip0pt\parsep0pt
\item
  Jede Schicht verfolgt einen bestimmten Zweck
\item
  Client-Server Modell
\end{itemize}

\paragraph{strict layering / closed
architecture}\label{strict-layering-closed-architecture}

Schichten dürfen nicht übersprungen werden und dürfen nur die direkt
darunter liegende Schicht aufrufen.

Vorteil: leichter test- und wartbar.

\paragraph{loose layering / open
architecture}\label{loose-layering-open-architecture}

Schichten dürfen übersprungen werden, höchste Schicht kann direkt mit
tiefster Schicht kommunizieren.

Vorteil: höhere Performanz.

\section{Pipes and Filters}\label{pipes-and-filters}

Filter modifizieren Daten, Pipes transportieren Daten. Problem:

\begin{itemize}
\itemsep1pt\parskip0pt\parsep0pt
\item
  Unterteilung Datenstrom in Einzelschritte
\item
  Einfach erweiterbar
\item
  Parallelverarbeitung gut möglich
\end{itemize}

Lösung:

\begin{itemize}
\itemsep1pt\parskip0pt\parsep0pt
\item
  Zerlegung in Einzelschritte
\item
  Jeder Verarbeitungsschritt ist ein Filter
\item
  Jeder Transportweg ist eine Pipe
\item
  Filter lesen, verarbeiten und geben Daten aus
\item
  Pipes transportieren und puffern Daten
\item
  Pipes entkoppeln Entitäten
\item
  Asynchrone Schreib- und Lesevorgänge möglich
\end{itemize}

\paragraph{Aktiver Filter}\label{aktiver-filter}

Eigenständig laufende Prozesse oder Threads, werden nach Instanziierung
von übergeordnetem Programmteil aufgerufen. Laufen kontinuierlich.

\paragraph{Passiver Filter}\label{passiver-filter}

Aufruf durch benachbartes Filterelement.

\paragraph{Push}\label{push}

Vorgänger hat Ausgabedaten erzeugt und ruft zu aktivierenden Filter auf.
(Daten werden zur Verfügung gestellt)

\paragraph{Pull}\label{pull}

Nachfolger fordert Daten an. (Daten werden benötigt)

\section{Plug-In}\label{plug-in}

Interface beschreibt Funktion, Plug-Ins sind die konkreten Umsetzungen
dieser Interfaces. Der Plugin-Manager sorgt für die korrekte Zuordnung
der konkreten Umsetzung zum Interface bei Aufruf.

Problem:

\begin{itemize}
\itemsep1pt\parskip0pt\parsep0pt
\item
  Erweiterbarkeit ohne Modifikation
\item
  Neue Funktionalität durch Dritte
\item
  Core-System funktioniert ohne Zusätze, bietet mit ihnen aber mehr
  Funktionalität
\end{itemize}

Lösung:

\begin{itemize}
\itemsep1pt\parskip0pt\parsep0pt
\item
  Interfaces für Erweiterungen
\item
  Plug-Ins implementieren diese
\item
  Plug-Ins ersetzen zur Laufzeit die Interfaces durch konkrete
  Implementierungen.
\end{itemize}

Beispiel: Interface: Escort-Dame, Konkrete Umsetzungen: Cindy, Mandy,
Scarlett, Plugin-Manager: Telefonistin der Agentur.

\section{Broker}\label{broker}

Broker ist verantwortlich für die Koordination von Kommunikation
(Weitergabe von Anfragen und Antworten)

Problem:

\begin{itemize}
\itemsep1pt\parskip0pt\parsep0pt
\item
  Skalierbarkeit
\item
  Verteilbarkeit
\item
  Lose Kopplung
\item
  Gegenseitige Abhängigkeit nicht sinnvoll
\end{itemize}

Lösung:

\begin{itemize}
\itemsep1pt\parskip0pt\parsep0pt
\item
  Komponenten klassifiziert durch deren Rolle
\item
  Serverkomponenten bieten n\textgreater{}=1 Services
\item
  Clients nutzen m\textgreater{}=1 Services
\item
  Rollen können sich ändern
\item
  Broker bietet Vermittlungs- und Zwischenschicht
\end{itemize}

\paragraph{marshalling}\label{marshalling}

Serialisierung von Methodenaufrufen für Netzwerktransmission

\paragraph{unmarshalling}\label{unmarshalling}

Rückkonvertierung in Standard-Methodenaufrufe auf Empfängerseite

Beispiel: Pornokino mit Gloryhole. Die Wand mit dem Loch ist der Broker,
die Kundschaft sind die Clients, die Dienstleisterinnen sind die für den
Client unzugänglichen Systeme und unterscheiden und/oder ergänzen sich
im Funktionsumfang.

\section{Service-Oriented Architecture
(SOA)}\label{service-oriented-architecture-soa}

\section{Model-View-Controller
(MVC)}\label{model-view-controller-mvc}

Problem: Mehrere Benutzeroberflächen, die sich schnell ändern können.
Dargestellte Informationen müssen stets aktuell sein.

Lösung: Aufgabenverteilung auf 3 Komponenten, Model verwaltet und verarbeitet
Daten, View stellt Daten dar, Controller verarbeitet Benutzereingaben

\paragraph{Push}\label{push-1}

Model sendet aktiv Daten an die View, welche dann die Darstellung
aktualisiert.

\paragraph{Pull}\label{pull-1}

View wird durch Model informiert, dass neue Daten vorhanden sind. Wird
die View \emph{immer} informiert ist es aktiver Pull, sonst passiver.
Der Controller wertet Benutzereingaben aus und aktualisiert ggf. das
Model oder die View. Die View zieht sich die neuen Daten nach
Information durch das Model.
