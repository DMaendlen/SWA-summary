\chapter{Beispielcode} 
 
 
\begin{minipage}[t]{0.45\linewidth}

	 \begin{lstlisting}[caption={Abstrakte Java Klasse}\label{lst:java-abstract-class},captionpos=none,language=JAVA]] 
	
public interface Geometrisch {

    public abstract double berechneFlaeche();
	
}



	\end{lstlisting}
	
	 \begin{lstlisting}[caption={Abstrakte Java Klasse}\label{lst:java-abstract-class},captionpos=none,language=JAVA]] 

public class Rechteck extends BasisFigur {

    public Rechteck(final double LAENGE, final double HOEHE) {
        super(LAENGE, HOEHE);
    }

    @Override
    public double berechneFlaeche() {
        return getLaenge()*getHoehe();
    }
}

	\end{lstlisting}	
	


\end{minipage}% <- sonst wird hier ein Leerzeichen eingefügt
\hfill
\begin{minipage}[t]{0.45\linewidth}
    

	\begin{lstlisting}[caption={Abstrakte Java Klasse}\label{lst:java-abstract-class},captionpos=none,language=JAVA]] 
public abstract class BasisFigur implements Geometrisch {

    double laenge;
    double hoehe;
	
    public BasisFigur(final double LAENGE, final double HOEHE) {
        laenge = LAENGE;
        hoehe = HOEHE;
    }

    public double getLaenge() {
        return laenge;
    }

    public void setLaenge(final double LAENGE) {
        laenge = LAENGE;
    }

    public double getHoehe() {
        return hoehe;
    }

    public void setHoehe(final double HOEHE) {
        hoehe = HOEHE;
    }
}
	\end{lstlisting}
\end{minipage}

\begin{minipage}[t]{0.45\linewidth}



\end{minipage}% <- sonst wird hier ein Leerzeichen eingefügt
\hfill
\begin{minipage}[t]{0.45\linewidth}
	 \begin{lstlisting}[caption={Abstrakte Java Klasse}\label{lst:java-abstract-class},captionpos=none,language=JAVA]]
public class RechtwinkligesDreieck extends BasisFigur {

    public RechtwinkligesDreieck(final double LAENGE, final double HOEHE) {
        super(LAENGE, HOEHE);
    }

    @Override
    public double berechneFlaeche() {
        return (getLaenge()*getHoehe())/2;
    }
}
	\end{lstlisting}
\end{minipage}
\hfill
\begin{minipage}[t]{0.45\linewidth}

	 \begin{lstlisting}[caption={Abstrakte Java Klasse}\laberl{lst:java-abstract-class},captionpos=none,language=JAVA]]
public class GeometrischerRechner {

    public static double berechneGesamtFlaeche(final Geometrisch[] f) {
        double ergebnis = 0;
        for (Geometrisch figur : f) {
            ergebnis += figur.berechneFlaeche();
        }
        return ergebnis;
    }
}
	\end{lstlisting}

\end{minipage}% <- sonst wird hier ein Leerzeichen eingefügt
\hfill
\begin{minipage}[t]{0.45\linewidth}
		 \begin{lstlisting}[caption={Abstrakte Java Klasse}\label{lst:java-abstract-class},captionpos=none,language=JAVA]] 
	 
public class GeometrischesFigurenBeispiel {

    public static void main(String[] args) {
        // 1.1. Erstellung einzelner Objekte
        Geometrisch rechteck = new Rechteck(10, 20);
        Geometrisch dreieck = new RechtwinkligesDreieck(10, 20);
		
        // 1.2. Separate Berechnung der Flaechen 
        System.out.println("Rechtecksflaeche: " + 
                           rechteck.berechneFlaeche() );
        System.out.println("Dreiecksflaeche: " + 
                           dreieck.berechneFlaeche() );
		
        // 2.1. Parameter fuer die uebergabe an GeometrischerRechner
        Geometrisch[] figuren = new Geometrisch[]{
                                          rechteck, dreieck};
		
        // 2.2. Berechnung und Ausgabe der Gesamtflaeche
        System.out.println("Gesamtflaeche: " + 
               GeometrischerRechner.berechneGesamtFlaeche( figuren ) );
        }
}  
    
	\end{lstlisting}
\end{minipage}