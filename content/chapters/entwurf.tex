\chapter{Entwurfsmuster}\label{entwurfsmuster}

\section{Zweck}\label{zweck}
Wiederverwendbare Vorlage zur Problemlösung, die in einem bestimmten
Zusammenhang einsetzbar ist.

\section{Abstraktion}\label{abstraktion}

\paragraph{Abstrakte Klasse}\label{abstrakte-klasse}

Wenn Methoden auf eine bestimmte Art implementiert werden müssen oder
wenn non-public Daten oder Methoden enthalten sein sollen. Dies
beeinträchtigt nicht die Möglichkeit, reine Methodenköpfe zu definieren,
die überschrieben werden müssen.

\paragraph{Interface}\label{interface}

Reine Schnittstellenbeschreibung.

\section{Erzeugungsmuster}\label{erzeugungsmuster}

\paragraph{Abstract Factory / Abstrakte Fabrik}\label{abstract-factory}

Schnittstelle zur Erzeugung einer Familie von Objekten, wobei die konkreten
Klassen der zu instanziierenden Objekte nicht näher festgelegt werden.

\paragraph{Factory / Fabrik}\label{factory}

Objekt, das durch Methodenaufruf andere Objekte erzeugt. Das zurückgegebene
Objekt ist neu und verweist nicht zwingend auf die Factory.

\paragraph{Object Pool}\label{object-pool}

Objekte, die mehrfach verwendet werden sollen, werden nur bei der ersten
Verwendung instanziiert und dann im Objektpool aufgehoben um sie
wiederverwenden zu können.

Vorteil: Reduzierter Aufwand (Zeit / Rechenleistung)

Nachteil: Erhöht Komplexität

\paragraph{Singleton}\label{singleton}

Stellt sicher, dass von einem Objekt nur eine Instanz existiert, ist in der
Regel global verfügbar.

\section{Strukturmuster}\label{strukturmuster}

\paragraph{Adapter}\label{adapter}

Übersetzt eine Schnittstelle in eine andere, dadurch wird die Kommunikation von
Klassen mit inkompatiblen Schnittstellen untereinander ermöglicht.

\paragraph{Bridge}\label{bridge}

Dient zur Trennung von Implementierung und ihrer Abstraktion. So können beide
unabhängig voneinander verändert werden.

\paragraph{Composition}\label{composition}

Repräsentiert Teil-Ganzes-Hierarchien, indem Objekte zu Baumstrukturen
zusammengefasst werden. Fasst in abstrakter Klasse sowohl primitive Objekte als
auch deren Behälter zusammen.

\paragraph{Decorator}\label{decorator}

Flexible Alternative zur Unterklassenbildung um eine Klasse nachträglich um
zusätzliche Funktionalitäten zu ergänzen.

\paragraph{Facade}\label{facade}

Bietet einheitliche, meist vereinfachte Schnittstelle zu einer Menge von
Schnittstellen eines Subsystems.

\paragraph{Proxy}\label{proxy}

Überträgt die Steuerung eines Objekts auf ein vorgelagertest
Stellvertreterobjekt.

\section{Verhaltensmuster}\label{verhaltensmuster}

\paragraph{Command}\label{command}

Kommandoobjekt kapselt einen Befehl um so zu ermöglichen, Objekte in eine
Warteschlange zu stellen, Logbucheinträge zu führen und Operationen rückgängig zu
machen.

\paragraph{Iterator}\label{iterator}

Stellt die Möglichkeit zur Verfügung, auf Elemente einer aggregierten Struktur
sequenziell zuzugreifen, ohne die Struktur zu enthüllen. Auch als Cursor
bekannt.

\paragraph{Mediator}\label{mediator}

Dienst zum Steuern des kooperativen Verhaltens von Objekten, wobei Objekte
nicht direkt kooperieren sondern über einen Vermittler.

\paragraph{Observer}\label{observer}

Beobachtetes Objekt hält eine Liste mit Beobachtern und informiert diese über
Veränderungen.

\paragraph{State}\label{state}

Kapselung zustandsabhängiger Verhaltensweisen von Objekten (vgl.
Zustandsautomat)

\paragraph{Strategy}\label{strategy}

Definiert eine Familie von Algorithmen, kapselt diese und macht sie
untereinander austauschbar. Lässt den Algorithmus unabhängig von den nutzenden
Clients variieren.


%\paragraph{Role}\label{role}
%
%\begin{figure}[h]
%	\includegraphics[width=0.3\textwidth]{images/role}
%\end{figure}

\paragraph{Visitor}\label{visitor}

Repräsentiert eine Operation, die auf Elemente einer Objektstruktur ausgeführt
wird. Ermöglicht die Definition einer neuen Operation ohne die Klassen der
Elemente zu ändern, auf die es ausgeführt wird.
